\documentclass[a4paper,12pt]{article}

% Sprachen und Zeichencodierung
\usepackage[utf8]{inputenc}
\usepackage[ngerman]{babel}

% Schrift und Typografie
\usepackage{lmodern}
\usepackage{setspace}
\usepackage{parskip}

% Seitenlayout und Struktur
\usepackage[a4paper,margin=2.5cm]{geometry}
\usepackage{titlesec}
\usepackage{float}
\usepackage{tocloft}

% Referenzen und Zitate
\usepackage[
    backend=bibtex,
    style=ieee,
    sorting=nyt,
    natbib=true,
    url=false,
    doi=true,
    eprint=false
]{biblatex}
\addbibresource{references.bib}

% Links
\usepackage[hidelinks]{hyperref}

% Abbildungen
\usepackage{graphicx}

% Nummerierung und Linien
\usepackage{lineno}

% Zitate und Sprachschnipsel
\usepackage{csquotes}

% Abkürzungen und Glossare
\usepackage[automake, acronym]{glossaries}
\makeglossaries

\newacronym{rpg}{RPG}{
engl. "role-playing games", sind ein Genre von Spielen,
in denen die Spieler in die Rolle eines imaginären Charakters schlüpfen
}

\newacronym{xp}{XP}{
engl. "experience points", sind ein Konzept in Spielen,
das den Fortschritt eines Charakters oder Spielers zeigt
}

\newacronym{api}{API}{
engl. "application programming interface", ist eine Sammlung von Definitionen
und Protokollen, die es Softwareanwendungen ermöglichen, miteinander zu
kommunizieren
}
% Einstellungen für das Inhaltsverzeichnis
\setcounter{tocdepth}{2}

% Absatzformatierung
\setlength{\parindent}{0pt}

\onehalfspacing

\begin{document}

% Titelseite
\pagenumbering{gobble}
\begin{titlepage}
	\centering
	{\LARGE Siemens Energy - BETI 2024 \par}
	\vspace{0.5cm}
	{\Large Realisierung eines Webauftritts \par}
	\vspace{3.5cm}
	{\huge\bfseries Socialmedia RPG \par}
	{\huge\bfseries Echo \par}
	\vspace{3cm}
	{\Large\itshape Emma Jammers \par}
	{\Large\itshape Ogulcan Kuecuek \par}
	{\Large\itshape Leon Woenckhaus \par}
	{\Large\itshape Nick Hildebrandt \par}
	{\Large\itshape Aaron Turyabahika \par}
	{\Large\itshape Andre Seiler \par}
	\vfill
	{\large \today\par}
\end{titlepage}
\newpage

% Inhaltsverzeichnis
\pagenumbering{roman}
\tableofcontents
\newpage

% Abbildungsverzeichnis
\listoffigures
\newpage

% Abkürzungsverzeichnis
\printglossary[type=\acronymtype, title=Abkürzungsverzeichnis]
\newpage

% Inhalt
\pagenumbering{arabic}
\setcounter{page}{1}

\section{Projektbeschreibung}
Das Social Media Projekt “Echo” ist ein
innovatives, kompetitives Social Media Netzwerk, das speziell für Gamer,
Digital Natives und Content Creator entwickelt wurde. Echo kombiniert die
Elemente traditioneller Social Media Plattformen wie Reddit mit
Rollenspiel-Mechaniken (\gls{rpg}), um ein dynamisches und interaktives
Nutzererlebnis zu schaffen.

Nutzer sammeln Erfahrungspunkte (\gls{xp}), indem sie Likes und Kommentare auf
ihre Posts von anderen Nutzern erhalten. Diese \gls{xp} sind ein Maß für die
Aktivität und Beliebtheit eines Nutzers innerhalb der Plattform. Zusätzlich zu
den \gls{xp} können Nutzer durch sogenannte “Streaks” weitere Erfahrungspunkte
sammeln. Ein Streak entsteht, wenn ein Nutzer über mehrere aufeinanderfolgende
Tage hinweg aktiv ist und regelmäßig Inhalte postet oder mit anderen
interagiert. Je länger der Streak, desto höher die Belohnung.

Die gesammelten Erfahrungspunkte ermöglichen es den Nutzern, ihr Profil und das
Design der Website individuell anzupassen. Dies umfasst personalisierte Themes,
exklusive Avatare und spezielle Layouts, die das Profil einzigartig machen.
Echo bietet verschiedene Themenbereiche, die den Nutzern nach Angabe ihrer
Interessen personalisierte Inhalte anzeigen. Dies stellt sicher, dass jeder
Nutzer relevante und ansprechende Inhalte sieht, die seinen Vorlieben
entsprechen.

Die Anzahl der gesammelten Erfahrungspunkte bestimmt die Relevanz und
Sichtbarkeit der eigenen Posts. Nutzer mit mehr \gls{xp} haben eine höhere
Wahrscheinlichkeit, dass ihre Beiträge von anderen gesehen werden. Im
Mittelpunkt von Echo stehen Gamification- Elemente, Gruppenzugehörigkeit und
das Belohnungsgefühl. Nutzer werden durch kontinuierliche Belohnungen und
Fortschritte motiviert, aktiv zu bleiben und sich in der Community zu
engagieren.

\subsection{Projektbegründung}
Warum ist das Projekt sinnvoll?

Was ist die Motivation hinter dem Projekt?

\subsection{Projektabgrenzung}
Was ist explizit nicht Teil des Projekts (insb.bei Teilprojekten)?

\newpage \section{Projektplanung}
In welchem Zeitraum und unter welchen
Rahmenbedingungen (z.B. Tagesarbeitszeit) findet das Projekt statt?

Verfeinerung der Zeitplanung, die bereits im Projektantrag vorgestellt wurde.

\subsection{Teamaufbau und Rollen}
\begin{enumerate}
    \item Projektmanagement
    \begin{itemize}
        \item Name
        \item Name
    \end{itemize}
    \item Backend-API
    \begin{itemize}
        \item Name
        \item Name
    \end{itemize}
    \item Frontend
    \begin{itemize}
        \item Name
        \item Name
    \end{itemize}
    \item Deplomay und integration
    \begin{itemize}
        \item Name
        \item Name
    \end{itemize}
\end{enumerate}

\subsection{Ressourcenplanung}
Detaillierte Planung der benötigten Ressourcen
(Hard-/Software, Räumlichkeiten usw.).

Ggfs. sind auch personelle Ressourcen einzuplanen (z.B. unterstützende
Mitarbeiter).

Hinweis: Häufig werden hier Ressourcen vergessen, die als selbstverständlich
ange- sehen werden (z.B. PC, Büro).

\subsection{Entwicklungsprozess}
Welcher Entwicklungsprozess wird bei der
Bearbeitung des Projekts verfolgt (z.B.  Wasserfall, agiler Prozess)?

\newpage \section{Zielplattform und Implementierung}
Beschreibung der Kriterien
zur Auswahl der Zielplattform (u.a. Programmiersprache, Datenbank,
Client/Server, Hardware).

\subsection{Architekturdesign}
Beschreibung und Begründung der gewählten
Anwendungsarchitektur (z.B. MVC).

Ggfs. Bewertung und Auswahl von verwendeten Frameworks sowie ggfs. eine kurze
Einführung in die Funktionsweise des verwendeten Frameworks.

\subsection{Benutzeroberfläche}
Entscheidung für die gewählte
Benutzeroberfläche (z.B. GUI, Webinterface).

Beschreibung des visuellen Entwurfs der konkreten Oberfläche (z.B. Mockups,
Menü- führung).

Ggfs. Erläuterung von angewendeten Richtlinien zur Usability und Verweis auf
Corpo- rate Design.

\subsection{Datenmodell}
Entwurf/Beschreibung der Datenstrukturen (z.B. ERM
und/oder Tabellenmodell, XML- Schemas) mit kurzer Beschreibung der wichtigsten
(!) verwendeten Entitäten.

Beschreibung der angelegten Datenbank (z.B. Generierung von SQL aus Modellie-
rungswerkzeug oder händisches Anlegen), XML-Schemas usw.

\subsection{Datenzugriff und Backend-API}
Zweck der Backend-API: Kurze Beschreibung, warum die API existiert und welche
Aufgabe sie im Projekt erfüllt.

Systemkontext: Darstellung, wie das Backend in die Gesamtarchitektur
eingebunden ist (z. B. Diagramm mit Datenbank, Frontend, API-Gateway).

Technologiestack: Beschreibung der eingesetzten Technologien (z. B.
Programmiersprache, Frameworks, Datenbank).

Designmuster: Falls zutreffend, z. B. REST, GraphQL, Microservices, etc.

Endpunkte: Liste der API-Endpunkte (z. B. GET /users, POST /orders).
Beschreibung des Zwecks jedes Endpunkts.

Request/Response-Formate: HTTP-Methoden (GET, POST, PUT, DELETE).
Beispielanfragen und -antworten (JSON, XML, etc.).  Fehlercodes und
Fehlermeldungen.

Authentifizierung und Autorisierung: Beschreibung des Sicherheitskonzepts (z. B.
OAuth, API-Keys, JWT).  Zugriffsbeschränkungen und Rollen.

\newpage \section{Abnahmephase}
Welche Tests (z.B. Unit-, Integrations-,
Systemtests) wurden durchgeführt und welche Ergebnisse haben sie geliefert
(z.B. Logs von Unit Tests, Testprotokolle der Anwen- der)?

\subsection{Bereitstellung}
Welche Schritte waren zum Deployment der Anwendung
nötig und wie wurden sie durchgeführt (automatisiert/manuell)?

\subsection{Fazit}
Wurde das Projektziel erreicht und wenn nein, warum nicht?

Ist der Auftraggeber mit dem Projektergebnis zufrieden und wenn nein, warum
nicht?

Wurde die Projektplanung (Zeit, Kosten, Personal, Sachmittel) eingehalten oder
haben sich Abweichungen ergeben und wenn ja, warum?

Hinweis: Die Projektplanung muss nicht strikt eingehalten werden. Vielmehr sind
Ab- weichungen sogar als normal anzusehen. Sie müssen nur vernünftig begründet
wer- den (z.B. durch Änderungen an den Anforderungen, unter-/überschätzter
Aufwand).

% Literaturverzeichnis
\newpage
\printbibliography

\end{document}
